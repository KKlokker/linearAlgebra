\documentclass[12pt, a4paper]{article}
\usepackage{caption}
\usepackage{graphicx}
\usepackage{hyperref}
\hypersetup{
    colorlinks,
    citecolor=black,
    filecolor=black,
    linkcolor=black,
    urlcolor=black
}
\usepackage{tikz-network}
\usepackage{amsmath, amsfonts, amssymb, amsthm}
\usepackage{algpseudocode}
\usepackage{algorithm}
\title{Linear algebra}
\date{2022}
\author{Kristoffer Klokker}

\usepackage{xcolor,listings}
\usepackage{textcomp}
\usepackage{color}
\usepackage{listings}
\definecolor{codegreen}{rgb}{0,0.6,0}
\definecolor{codegray}{rgb}{0.5,0.5,0.5}
\definecolor{codepurple}{HTML}{C42043}
\definecolor{backcolour}{HTML}{F2F2F2}
\definecolor{bookColor}{cmyk}{0,0,0,0.90}  
\color{bookColor}

\lstset{upquote=true}

\lstdefinestyle{mystyle}{
    backgroundcolor=\color{backcolour},   
    commentstyle=\color{codegreen},
    keywordstyle=\color{codepurple},
    numberstyle=\numberstyle,
    stringstyle=\color{codepurple},
    basicstyle=\footnotesize\ttfamily,
    breakatwhitespace=false,
    breaklines=true,
    captionpos=b,
    keepspaces=true,
    numbers=left,
    numbersep=10pt,
    showspaces=false,
    showstringspaces=false,
    showtabs=false,
    tabsize=3,
}
\lstset{style=mystyle}
\usepackage{zref-base}

\makeatletter
\newcounter{mylstlisting}
\newcounter{mylstlines}
\lst@AddToHook{PreSet}{%
  \stepcounter{mylstlisting}%
  \ifnum\mylstlines=1\relax
    \lstset{numbers=none}
  \else
    \lstset{numbers=left}
  \fi
  \setcounter{mylstlines}{0}%
}
\lst@AddToHook{EveryPar}{%
  \stepcounter{mylstlines}%
}
\lst@AddToHook{ExitVars}{%
  \begingroup
    \zref@wrapper@immediate{%
      \zref@setcurrent{default}{\the\value{mylstlines}}%
      \zref@labelbyprops{mylstlines\the\value{mylstlisting}}{default}%
    }%
  \endgroup
}

% \mylstlines print number of lines inside listing caption
\newcommand*{\mylstlines}{%
  \zref@extractdefault{mylstlines\the\value{mylstlisting}}{default}{0}%
}
\makeatother


\newcommand\numberstyle[1]{%
    \footnotesize
    \color{codegray}%
    \ttfamily
    \ifnum#1<10 0\fi#1 |%
}


\begin{document}
	\maketitle
	\clearpage
	\tableofcontents
	\clearpage
	\section{Systems of linear Systems of Linear Equations}
		A system of linear equations are multiple equations containing unknows which are shared. Ex.
		\begin{alignat*}{4}
		   2x & {}+{} &  y & {}+{} & 3z & {}={} & 10 \\
		    x & {}+{} &  y & {}+{} &  z & {}={} &  6 \\
		    x & {}+{} & 3y & {}+{} & 2z & {}={} & 13
		\end{alignat*}
		The same system may be wirtten in form of a matrix\\
		$$\begin{bmatrix}
			2 & 1 & 3 & 10\\
			1 & 1 & 1 & 6\\
			1 & 3 & 2 & 13
		\end{bmatrix}$$
		When solving a system their may be
		\begin{itemize}
			\item No solutions - No possible value can be assigned to the variable such all equation are true
			\item One solution - A combination of values can be assigned to make every equation true
			\item Infinite solutions - One or more unknows may have an infinite amount of possible assignable values
		\end{itemize}
		By transforming a system of equations to a matrice, the following operations can be performed:
		\begin{itemize}
			\item Multiply a row through by a nonzero constant.
			\item Interchange two rows.
			\item Add a constant times one row to another.
		\end{itemize}
		\subsection{Gaussian Elimination}
			Solving a system of equation in a matrice, can be done such the matrice has the following requirements
			\begin{itemize}
				\item If the row contain nothing but zeroes the first number should be a 1, called the leading 1.
				\item If a row is made of nothing but zeroes it should be grouped at the bottom
				\item In two rows the top row should contain a leading 1 further to the left than the bottom
				\item Each column which contains a leading 1, every number in the same column below should be 0
			\end{itemize}
			This form is called row echelon form.\\
			The solution may also be written as:
			$$x_1=4,x_2=6,x_3=t,x_4=v,x_5=1$$
			Where variables means they can be any possible asignment or a function with given restrictions.\\
			In case of the leading 1 column is zero both above and underneath the matrice is in reduced echlon form.\\[4mm]
			To make a matrice into echoleon form the following algorithm can be used:\\
			\begin{enumerate}
				\item Locate the leftmost column that does not consist entruely of zeros and exhange it to the top
				\item Multiply the top row by $\frac{1}{a}$ where $a$ is the leading number in the row
				\item subtract top row from every other below row, such the top row is the only non zero value in the column
				\item Repeat 2 and 3 but ignore the top row and let the second top row be the top
			\end{enumerate}
			To reduce the echloen form, from the bottom the bottom row is added to the above rows until the leading 1 is the only in the column. This is repeated until the top is reached.\\
			A homogeneous linear system are systems which all constant (right part of equal) are 0 and the trivial solution (all variables are assigned 0) are a possible solution.\\
			A free variable is the term for a variable which can be assigned multiple values. The number of free variables wil lbe equal to the number of variables minus zero rows.\\
			In a homogenous linear system if the number of unknows exceed the number of equation there will be an infinite amount of solutions.\\
			Back substitution is a method taking the echoloen form and starting from the bottom isolating the leading variable and substituting it upwards.\\
			A echoloen form is not unique to a system but a reduced echoloen form is unique and the number of zero rows will be unique.
			
\end{document}
