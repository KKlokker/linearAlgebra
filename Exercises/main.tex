\documentclass[12pt, a4paper]{article}
\usepackage{caption}
\usepackage{graphicx}
\usepackage{hyperref}
\hypersetup{
    colorlinks,
    citecolor=black,
    filecolor=black,
    linkcolor=black,
    urlcolor=black
}
\usepackage{tikz-network}
\usepackage{amsmath, amsfonts, amssymb, amsthm}
\usepackage{algpseudocode}
\usepackage{algorithm}
\title{Linear algebra\\ Exercises}
\date{2022}
\author{Kristoffer Klokker}

\usepackage{xcolor,listings}
\usepackage{textcomp}
\usepackage{color}
\usepackage{listings}
\definecolor{codegreen}{rgb}{0,0.6,0}
\definecolor{codegray}{rgb}{0.5,0.5,0.5}
\definecolor{codepurple}{HTML}{C42043}
\definecolor{backcolour}{HTML}{F2F2F2}
\definecolor{bookColor}{cmyk}{0,0,0,0.90}  
\color{bookColor}

\lstset{upquote=true}

\lstdefinestyle{mystyle}{
    backgroundcolor=\color{backcolour},   
    commentstyle=\color{codegreen},
    keywordstyle=\color{codepurple},
    numberstyle=\numberstyle,
    stringstyle=\color{codepurple},
    basicstyle=\footnotesize\ttfamily,
    breakatwhitespace=false,
    breaklines=true,
    captionpos=b,
    keepspaces=true,
    numbers=left,
    numbersep=10pt,
    showspaces=false,
    showstringspaces=false,
    showtabs=false,
    tabsize=3,
}
\lstset{style=mystyle}
\usepackage{zref-base}

\makeatletter
\newcounter{mylstlisting}
\newcounter{mylstlines}
\lst@AddToHook{PreSet}{%
  \stepcounter{mylstlisting}%
  \ifnum\mylstlines=1\relax
    \lstset{numbers=none}
  \else
    \lstset{numbers=left}
  \fi
  \setcounter{mylstlines}{0}%
}
\lst@AddToHook{EveryPar}{%
  \stepcounter{mylstlines}%
}
\lst@AddToHook{ExitVars}{%
  \begingroup
    \zref@wrapper@immediate{%
      \zref@setcurrent{default}{\the\value{mylstlines}}%
      \zref@labelbyprops{mylstlines\the\value{mylstlisting}}{default}%
    }%
  \endgroup
}

% \mylstlines print number of lines inside listing caption
\newcommand*{\mylstlines}{%
  \zref@extractdefault{mylstlines\the\value{mylstlisting}}{default}{0}%
}
\makeatother


\newcommand\numberstyle[1]{%
    \footnotesize
    \color{codegray}%
    \ttfamily
    \ifnum#1<10 0\fi#1 |%
}


\begin{document}
	\maketitle
	\clearpage
	\tableofcontents
	\clearpage
	\section{Week 36}
		\subsection{Which equation are linear in $x_1$, $x_2$ and $x_3$}
			\subsubsection{$x_1+5x_2-\sqrt{2}x^3=1$}
				This is a linear equation
			\subsubsection{$x_1=-7x_2+3x_3$}
				This is a linear equation
			\subsubsection{$x_1^{3/5}-2x_2+x_3=4$}
				This is not a linear equation with $x_1$ having a power
		\subsection{Convert from matrix form to equation form}
			$\begin{bmatrix}
				2 & 0 & 0 \\
				3 & -4 & 0\\
				0 & 1 & 1 
			\end{bmatrix}$\\
			$2x_1=0$\\
			$3x_1-4x_2=0$\\
			$x_2=1$
		\subsection{Convert from equations to matrix}
			$-6x_1-x_2+3x_3=4$\\
			$5x_2-x_3=1$\\
			$\begin{bmatrix}
				-6 & -1 & 3 & 4 \\
				0 & 5 & -1 & 1
			\end{bmatrix}$
		\subsection{Determine if the solution hold in the following system}
			$(5,8,1)$\\
			$x+2y-2z=3$\\
			$3x-y+z=1$\\
			$-x+5y-5z=5$\\[4mm]
			$5+2(8)-2(1)=3$\\
			$19=3$\\
			By the first equation the solution does not hold
		\subsection{Determine if the following matrices are in echoleon form or reduced echelon form}
			$\begin{bmatrix}
				1 & 0 & 0 \\
				0 & 1 & 0\\
				0 & 0 & 1 
			\end{bmatrix}$\\
			Reduced echelon form\\
			$\begin{bmatrix}
				1 & -3 & 4 & 7 \\
				0 & 1 & 2 & 2\\
				0 & 0 & 1  & 5
			\end{bmatrix}$\\
			Echelon form
	\section{Week 38}
		\subsection{Show that the determinant og the matix is 0}
			$\begin{bmatrix}
				-2 & 8 & 1 & 4 \\
				3 & 2 & 5 & 1\\
				1 & 10 & 6  & 5\\
				4 & -6 & 4  & -3
			\end{bmatrix}$\\[3mm]
			Column 2 and 4 are proportional to eachother therefore making the det=0
		\subsection{Is the following matrix invertible}
			$\begin{bmatrix}
				2 & 0 & 3 \\
				0 & 3 & 2 \\
				-2 & 0 & -4
			\end{bmatrix}$\\[3mm]
			The determinant is -6 and therefore not zero and therefore invertible
		\subsection{Find the standard matrix for the transformation defined by the equations}
			\begin{align*}
				w_1=7x_1+2x_2-8x_3\\
				w_2=-x_2+5x_3\\
				w_3=4x_1+7x_2-x_3
			\end{align*}
			$\begin{bmatrix}
				7 & 2 & -8 \\
				0 & -1 & 5 \\
				4 & 7 & -1
			\end{bmatrix}$\\[3mm]
		\subsection{Is a linear function a transformation of $R$}
			Yes the linear function can be a transform of the space $R$
		\subsection{The images of the standard basis vectors for $R^3$ are given for a linear trasnformation $T:R^3\rightarrow R^3$. Find the standard matrix for the transformation and find $T(x)$}
			\begin{align*}
					T(e_1)=
				\begin{bmatrix}
					1 \\
					3 \\
					0
				\end{bmatrix}\\[3mm]
					T(e_2)=
				\begin{bmatrix}
					0 \\
					0 \\
					1
				\end{bmatrix}\\[3mm]
					T(e_3)=
				\begin{bmatrix}
					4 \\
					-3 \\
					-1
				\end{bmatrix}\\[3mm]
					x=
				\begin{bmatrix}
					2 \\
					1 \\
					0
				\end{bmatrix}\\[3mm]
					A=
				\begin{bmatrix}
					1 & 0 & 4 \\
					3 & 0 & -3\\
					0 & 1 & -1
				\end{bmatrix}\\[3mm]
					T(x)=
				\begin{bmatrix}
					2 \\
					6 \\
					1
				\end{bmatrix}\\[3mm]
			\end{align*}
		\subsection{Find the standard matrix $A$ for the linear transformation $T:R^2\rightarrow R^2$ for which}
			$$T(\begin{bmatrix}
					1 \\
					1
				\end{bmatrix})=\begin{bmatrix}
					1 \\
					-2
				\end{bmatrix},T(\begin{bmatrix}
					2 \\
					3
				\end{bmatrix})=\begin{bmatrix}
					-2 \\
					5
				\end{bmatrix}$$
				
				
			\begin{align*}
				\begin{bmatrix}
					1 \\
					0
				\end{bmatrix}=c_1\begin{bmatrix}
					1 \\
					1
				\end{bmatrix}+c_2\begin{bmatrix}
					2 \\
					3
				\end{bmatrix}\\
				\begin{bmatrix}
					0 \\
					1
				\end{bmatrix}=k_1\begin{bmatrix}
					1 \\
					1
				\end{bmatrix}+k_2\begin{bmatrix}
					2 \\
					3
				\end{bmatrix}\\
				\text{c variable found by}\\
				\begin{bmatrix}
					1 & 2 & 1 \\
					1 & 3 & 0
				\end{bmatrix}\\
				\text{k variable found by}\\
				\begin{bmatrix}
					1 & 2 & 0 \\
					1 & 3 & 1
				\end{bmatrix}\\
				\text{Gauss jordian magic!}\\
				c_1=3, c_2=-1, k_1=-2,k_2=1\\
				T(
				\begin{bmatrix}
					1 \\
					0
				\end{bmatrix})=3T(
				\begin{bmatrix}
					1 \\
					1
				\end{bmatrix})+(-1)T(
				\begin{bmatrix}
					2 \\
					3
				\end{bmatrix})\\
				=\begin{bmatrix}
					3 \\
					-6
				\end{bmatrix}-
				\begin{bmatrix}
					-2 \\
					5
				\end{bmatrix}=\begin{bmatrix}
					5 \\
					-11
				\end{bmatrix}\\
				T(
				\begin{bmatrix}
					0 \\
					1
				\end{bmatrix})=-2T(
				\begin{bmatrix}
					1 \\
					1
				\end{bmatrix})+1T(
				\begin{bmatrix}
					2 \\
					3
				\end{bmatrix})\\
				=\begin{bmatrix}
					-2 \\
					4
				\end{bmatrix}+
				\begin{bmatrix}
					-2 \\
					5
				\end{bmatrix}=\begin{bmatrix}
					-4 \\
					9
				\end{bmatrix}\\[4mm]
				A=\begin{bmatrix}
				5&-4\\
				-11 & 9
					\end{bmatrix}
			\end{align*}
	\section{Week 39}
		\subsection{Determine if the set is a vector space}
			\subsubsection{$n$-tuples of real numbers that have the form $(x,x,...,x)$ with the standard opeations $R^n$}
				\begin{itemize}
					\item $u,v\in V \rightarrow v+u\in V$ True
					\item $x+v=v+u$ - True
					\item $u+(v+w)=(u+v)+w$ - True
					\item $The zero vector exists$ - True (0,0,0,0)
					\item There exists a $u$ in $V$ such $u+(-u)=0$ - True
					\item $uk\in V$ where $k$ is a scalar - True
					\item $k(u+v)=kv+ku$ - True
					\item $k(mu)=(km)u$ - True
					\item $1u=u$ - True
				\end{itemize}
			\subsubsection{The set of matrices in the form $\begin{bmatrix}
				a&0\\
				0 & b
					\end{bmatrix}$}
				\begin{itemize}
					\item $u,v\in V \rightarrow v+u\in V$ True
					\item $x+v=v+u$ - True
					\item $u+(v+w)=(u+v)+w$ - True
					\item $The zero vector exists$ - True a=0 and b=0
					\item There exists a $u$ in $V$ such $u+(-u)=0$ - True
					\item $uk\in V$ where $k$ is a scalar - True
					\item $k(u+v)=kv+ku$ - True
					\item $k(mu)=(km)u$ - True
					\item $1u=u$ - True
				\end{itemize}
		\subsection{Use the subspace test to determine if it a subspace of $R^3$}
			\begin{itemize}
				\item $(a,0,0)$ - $(a_1,0,0)+(a_2,0,0)=(a_3,0,0)$ and $k(a_1,0,0)=(a_2,0,0)$, therefore it is a subspace
				\item $(a,1,1)$ - $(a_1,0,0)+(a_2,1,1=(a_3,2,2$ which is in not in the space and therefore not a subspace
				\item $(a,b,c)$ where $b=a+c$ - $(a_1,b_1,c_1)+(a_2,b_2,c_2)=(a_1+a_2,a_1+a_2+c_1+c_2,c_1+c_2)$ and $k(a_1,b_1,c_1)=(ka_1,k(a_1+c_1),c_1)$ which holds true and therefore is asubspace
			\end{itemize}
		\subsection{Use the subspace test to determine if it a subspace of $F(-\infty,\infty)$}
			\begin{itemize}
				\item All functions $f$ in $F(-\infty,\infty)$ for which $f(0)=0$ - $f_1(0)+f_2(0)=0$ and $k\cdot f_1(0)=k\cdot 0=0$ therfore it is a subspace
				\item All functions $f$ in $F(-\infty,\infty)$ for which $f(0)=1$ - $f_1(0)+f_2(0)=2$ therefore making it not a subspace
			\end{itemize}
		\subsection{Use the subspace test to determine the a bounded set of sequence of real number is a subspace of the sequence of real number in an infinite tuple}
			The operations are defined as\\
			$(a_n)_{n\in N}+(b_n)_{n\in N}:=(a_n+b_n)_{n\in N}$\\
			$k\cdot (a_n)_{n\in N}:=(ka_n)_{n}$\\
			The sequence is bounded by a $C$ value such $-C\leq a_n \leq C$.\\
			The bounded sequence will be a subspace if we defined $a_n+b_n\leq C$ and $ka_n\leq C$
		\subsection{Verify that $\mathbb{M}_{m,n}(\mathbb{R})$ is a vector space}
				\begin{itemize}
					\item $u,v\in V \rightarrow v+u\in V$ True
					\item $x+v=v+u$ - True
					\item $u+(v+w)=(u+v)+w$ - True
					\item $The zero vector exists$ - True everything in the matrix equal 0
					\item There exists a $u$ in $V$ such $u+(-u)=0$ - True
					\item $uk\in V$ where $k$ is a scalar - True
					\item $k(u+v)=kv+ku$ - True
					\item $k(mu)=(km)u$ - True in the case that the size of matrices unlike $k_{2,1},u_{2,2},m_{2,1}$
					\item $1u=u$ - True
				\end{itemize}
		\subsection{Which if the following are linear combination of $u=(0,-2,2)$ and $v=(1,3,-1)$}
			\subsubsection{$(2,2,2)$}
				\begin{align*}
					(2,2,2)&=k_1(0,-2,2)+k_2(1,3,-1)\\
					&=(k_1+k_2,-2k_1+3k_2,2k_1-k_2)\\[4mm]
					k_1+k_2&=2\\
					-2k_1+3k_2&=2\\
					2k_1-k_2&=2\\[4mm]
					k_1&=0.8\\
					k_2&=1.2
				\end{align*}
				It can be seen that the third condition is not met and therefore it is not a linear combination
			\subsubsection{$(0,4,5)$}
				\begin{align*}
					(0,4,5)&=k_1(0,-2,2)+k_2(1,3,-1)\\
					&=(k_1+k_2,-2k_1+3k_2,2k_1-k_2)\\[4mm]
					k_1+k_2&=0\\
					-2k_1+3k_2&=4\\
					2k_1-k_2&=5\\[4mm]
					k_1&=-0.8\\
					k_2&=0.8
				\end{align*}
				It can be seen that the second equation is not met and therefore it is not a linear combination
			\subsubsection{$(0,0,0)$}
				\begin{align*}
					(0,4,5)&=k_1(0,-2,2)+k_2(1,3,-1)\\
					&=(k_1+k_2,-2k_1+3k_2,2k_1-k_2)\\[4mm]
					k_1+k_2&=0\\
					-2k_1+3k_2&=0\\
					2k_1-k_2&=0\\[4mm]
					k_1&=0\\
					k_2&=0
				\end{align*}
				This is a linear combination and by definition origo will always be a linear combination since a linear transformation does not move it.
		\subsection{Explain why the following form linearly dependent sets of vectors}
			\begin{itemize}
				\item $u_1=(-1,2,4)$ and $u_2=(5,-10,-20)$ in $\mathbb{R}^3$ - Vector $u_2$ is a scalar multiple of $-5u_1$
				\item $u_1=(3,-1),u_2=(4,5),u_3=(-4,7)$ in $\mathbb{R}^2$ - A linearly independent system would only need 2 vectors therefore making it dependent
			\end{itemize}
		\subsection{Determien whether the vectors are linearly independent in $R^3$}
			\begin{itemize}
				\item $(3,8,7),(1,5,3,7),(2,-1,2,6),(4,2,6,4)$ - Each vector is a none scalar of each other such the zero test can only be performed with only 0 coefficients
				\item $(3,0,-3,6),(0,2,3,1),(0,-2,-2,0),(-2,1,2,1)$ - Each vector is a none scalar and by the zero test shows the only solution is 0 coefficients
			\end{itemize}
		\subsection{Show that the following set of vectors forms a basis for $R^2$}
			$\{(2,1),(3,0)\}$\\
			It can be seen that the determinant of $\begin{bmatrix} 2 & 3 \\ 1 & 0 \end{bmatrix}$ is -3 and therefore not 0 and therefore the rows are linearly independent
		\subsection{Show that the following polynomials form a basis for $P_2$}
			$\{x^2+1,x^{-2}-1, 2x-1\}$\\
			It can be seen that the determinant of $\begin{bmatrix} 1 & -1 & -1 \\ 0 & 0 & 2\\ 1 & 1 & 0 \end{bmatrix}$ is -4 and therefore not 0 and therefore the column vectors span the space
		\subsection{Find the coordinate vector of $v$ relative to the basis $S=\{v_1,v_2,v_3\}$ for $R^3$}
			$$v=(2,-1,3);v_1=(1,0,0), v_2=(2,2,0),v_3=(3,3,3)$$
			$\begin{bmatrix} 1 & 0 & 0 \\ 2 & 2 & 0 \\ 3 & 3 & 3 \end{bmatrix}\cdot \begin{bmatrix} 2\\-1\\3 \end{bmatrix}=\begin{bmatrix} 2\\2\\12 \end{bmatrix}$
		\subsection{In each part let $T_a:R^3\rightarrow R^3$ be multiplication by $A$, and let $\{e_1,e_2,e_3\}$ be the standard basis for $R^3$. Determine whether the set $\{T_A(e_1),T_A(e_2),T_A(e_1),\}$ is linearly independent in $R^2$}
			\subsubsection{$A=\begin{bmatrix} 1 & 1 & 1 \\ 0 & 1 & -3\\ -1 & 2 & 0\end{bmatrix}$}
				Since the determinant of $A$ is not zero means that $\{T_A(e_1),T_A(e_2),T_A(e_1),\}$ is linearly independent
			\subsubsection{$A=\begin{bmatrix} 1 & 1 & 2 \\ 0 & 1 & 1\\ 0 & -1 & 2\end{bmatrix}$}
				Since the determinant of $A$ is zero means that $\{T_A(e_1),T_A(e_2),T_A(e_1),\}$ is linearly dependent
		\subsection{In each part, find a basis for the given subspace of $R^3$, and state its dimension.}
			\subsubsection{The plane $3x-2y+5z=0$}
				\begin{align*}
					y&=1.5x+2.5z\\
					\begin{bmatrix} x \\ y\\ z\\\end{bmatrix}&=\begin{bmatrix}x\\1.5x+2.5z\\z\end{bmatrix}\\
					&=x\begin{bmatrix}1\\1.5\\0\end{bmatrix}+z\begin{bmatrix}0\\2.5\\1\end{bmatrix}\\
					&\{\begin{bmatrix}1\\1.5\\0\end{bmatrix},\begin{bmatrix}0\\2.5\\1\end{bmatrix}\}
				\end{align*}
				Since the matrices are non zero and linearly independent they are the basis.
			\subsubsection{The plane $x-y=0$}
				\begin{align*}
					x=y\\
					\begin{bmatrix} x \\ y\\ \end{bmatrix}&=\begin{bmatrix}y\\y\end{bmatrix}\\
					&=y\begin{bmatrix}1\\1\end{bmatrix}\\
					&\{\begin{bmatrix}1\\1\end{bmatrix}\}
				\end{align*}
				
			\subsubsection{The line $x=2t,y=-t,z=4t$}
				\begin{align*}
					y&=-0.5x\\
					z&=2x\\
					\begin{bmatrix} x \\ y\\ z\\\end{bmatrix}&=\begin{bmatrix}x\\-0.5x\\2x\end{bmatrix}\\
					&=x\begin{bmatrix}1\\-0.5\\2\end{bmatrix}\\
					&\{\begin{bmatrix}1\\-0.5\\2\end{bmatrix}\}
				\end{align*}
			
			\subsubsection{All vectors of the form $(a,b,c)$, where $b=a+c$}
				\begin{align*}
					b&=a+c\\
					\begin{bmatrix} a \\ b\\ c\\\end{bmatrix}&=\begin{bmatrix}a\\a+c\\c\end{bmatrix}\\
					&=a\begin{bmatrix}1\\1\\0\end{bmatrix}+c\begin{bmatrix}0\\1\\1\end{bmatrix}\\
					&\{\begin{bmatrix}1\\1\\0\end{bmatrix},\begin{bmatrix}0\\1\\1\end{bmatrix}\}
				\end{align*}
				Since the matrices are non zero and linearly independent they are the basis.
		\subsection{Find a standard basis vector for $R^3$ that can be added to the set, to produce a bassi for $R^3$ / Enlarge to vector set to a basis}
			\subsubsection{$v_1=(-1,2,3),v_2=(1,-2,-2)$}
				\begin{align*}
					det(\begin{bmatrix} -1&2&3\\1&-2&-2\\a&b&c\end{bmatrix})&=-1\cdot det(\begin{bmatrix}2&3\\b&c\end{bmatrix})-2\cdot det(\begin{bmatrix}1&-2\\a&c\end{bmatrix}+3\cdot det(\begin{bmatrix}1&-2\\a&b\end{bmatrix})\\
					&=-(2c-3b)-2(1c-(-2a))+3(ab-(-2a))\\
					&=2c+3b-2c-4a+3ab-2a\\
					&=3b-6a+3ab\\
					0&\neq 3b-6a+3ab
				\end{align*}
				Therefore the missing vector in the form $(a,b,c)$, $c$ can be any value and $a$ and $b$ must values such the last statement holds true. Otherwise the determinant will be 0 and the vectors would be linearly dependent.
			\subsubsection{$v_1=(1,-1,0),v_2=(3,1,-2)$}
				\begin{align*}
					det(\begin{bmatrix} 1&-1&0\\3&1&-2\\a&b&c\end{bmatrix})&=1\cdot det(\begin{bmatrix}1&-2\\b&c\end{bmatrix})-(-1)\cdot det(\begin{bmatrix}3&-2\\a&c\end{bmatrix}+0\cdot det(\begin{bmatrix}3&1\\a&b\end{bmatrix})\\
					&=1(1c-2b)+1(3c-(-2a))\\
					&=c-2b+3c+2a\\
					0&\neq 2a+2b+2c\\
					2a&\neq -2b-2c\\
					a&\neq -b-c
				\end{align*}
				Therefor in the missing vector in the form $(a,b,c)$ it must hold true that $a$ is not equal to $-b-c$. Otherwise the determinant will be 0 and the vectors would be linearly dependent.
		\subsection{Find a bassi for the subspace of $R^3$ that is spanned by the vectors $v_1=(1,0,0),v_2=(1,0,1),v_3=(2,0,1),v_4=(3,3,3,4)$}
		By setting it up in a matrix and using gauss elimination the basis vectors can be found
			$(1,0,0),(0,0,1)$
		\subsection{Consider the bases $B=\{u_1,u_2\}$ and $B'=\{u_1',u_2'\}$ for $R^2$ NOT DONE}
			$$u_1=\begin{bmatrix}2\\2\end{bmatrix},u_2=\begin{bmatrix}4\\-1\end{bmatrix},u_1'=\begin{bmatrix}1\\3\end{bmatrix},u_2'=\begin{bmatrix}-1\\-1\end{bmatrix}$$
			\subsubsection{Find the transmission matrix from $B$ to $B'$}
				[new basis | old basis]
				$$\begin{bmatrix}1 & -1 & 2 & 4\\3 & -1 & 2 & 1 \end{bmatrix}$$
				Reducing gives
				$$\begin{bmatrix}1 & 0 & 0 & -1 \\ 0 & 1 & -2 & -5 \end{bmatrix}$$
				Therefore making the transistion matrix
				$$\begin{bmatrix} 0 & -1 \\ -2 & -5 \end{bmatrix}$$
			\subsubsection{Find the transmission matrix from $B$ to $B'$}
				[new basis | old basis]
				$$\begin{bmatrix} 2 & 4 & 1 & -1 \\ 2 & 1 &3 & -1  \end{bmatrix}$$
				Reducing gives
				$$\begin{bmatrix}1 & 0 & 1 & 0 \\0 & 1 & 0 & 0 \end{bmatrix}$$
				Therefore making the transistion matrix
				$$\begin{bmatrix} 0 & 1 \\0 & 0 \end{bmatrix}$$
			\subsubsection{Calculate the vector in both spaces}
				$$w=\begin{bmatrix}3\\-5\end{bmatrix}$$
				\begin{align*}
					[w]_b=B\cdot w\\
					[w]_b=3\begin{bmatrix}2 \\2 \end{bmatrix}-5\begin{bmatrix}4 \\-1 \end{bmatrix}\\
					[w]_b=\begin{bmatrix}-14\\11\end{bmatrix}\\
					[w]_{b'}= \begin{bmatrix} 0 & -1 \\ -2 & 5 \end{bmatrix}\cdot [w]_b\\
					[w]_{b'}=-14\begin{bmatrix} 0 \\ -2  \end{bmatrix}+11\begin{bmatrix} -1 \\  5 \end{bmatrix}\\
					[w]_{b'}=\begin{bmatrix} -11\\ 37 \end{bmatrix}\\
					[w]_{b'}=3\begin{bmatrix} 1 \\ 3  \end{bmatrix}-5\begin{bmatrix} -1 \\-1 \end{bmatrix}\\
					[w]_{b'}=\begin{bmatrix} -2\\ 4\end{bmatrix}
				\end{align*}
		\subsection{Let $V$ be the space spanned by $f_1=\sin x$ and $f_2=\cos x$}
			\subsubsection{Show that $g_1=2\sin x + \cos x$ and $g_2=3\cos x$ form a basis for $V$}
				It can be seen that since one consist
			\subsubsection{Find the transition matrix from $B=\{f_1,f_2\}$ to $B'=\{g_1,g_2\}$}
				\begin{align*}
					x_1\cdot (2\sin x + \cos x)+x_3\cdot 3\cos x = \sin x\\
					x_2\cdot (2\sin x + \cos x)+x_4\cdot 3\cos x = \cos x\\
					x_1=\frac{1}{2}\\
					x_2=-\frac{1}{6}\\
					x_3=0\\
					x_4=\frac{1}{3}\\
					\begin{bmatrix}\frac{1}{2} & 0\\ -\frac{1}{6} & \frac{1}{3}\end{bmatrix}
				\end{align*}
			\subsubsection{Find the transition matrix from $B'=\{g_1,g_2\}$ to $B=\{f_1,f_2\}$}
				\begin{align*}
					x_1\cdot (\sin x)+x_3\cdot \cos x = 2\sin x + \cos x\\
					x_2\cdot (\sin x)+x_4\cdot \cos x = 3\cos x\\
					x_1=2\\
					x_2=0\\
					x_3=1\\
					x_4=3\\
					\begin{bmatrix} 2 & 1 \\ 0 & 3\end{bmatrix}
				\end{align*}
			\subsubsection{Find $h$ in the two basis}
				\begin{align*}
					h=\begin{bmatrix}2\sin x \\- 5\cos x\end{bmatrix}\\
					[h]_B=\begin{bmatrix}\sin x\\\cos x\end{bmatrix}\cdot \begin{bmatrix}x_1\\x_2\end{bmatrix}=h\\
					x_1=2\\
					x_2=-5\\
					[h]_B=\begin{bmatrix}2\\-5\end{bmatrix}\\
					[h]_{B'}=\begin{bmatrix}\frac{1}{2} & 0\\ -\frac{1}{6} & \frac{1}{3}\end{bmatrix}\cdot \begin{bmatrix}2\\-5\end{bmatrix}
					[h]_{B'}=\begin{bmatrix}2\cdot \frac{1}{2}-5\cdot 0\\2\cdot (-\frac{1}{6})-5\cdot \frac{1}{3}\end{bmatrix}\\
					[h]_{B'}=\begin{bmatrix}1\\-2\end{bmatrix}\\[4mm]
					[h]_{B'}=\begin{bmatrix}2\sin x + \cos x\\3\cos x\end{bmatrix}\cdot \begin{bmatrix}x_1\\x_2\end{bmatrix}=h\\
					x_1=1\\
					x_2=-2
				\end{align*}
		\subsection{Which of these are linear transistions and what is their kernel}
			\begin{itemize}
				\item $T(A)=A^2$ - Not linear
				\item $T(A)=tr(A)$ - all matrices in the form $\begin{bmatrix}a & b \\ c & -a\end{bmatrix}$
				\item $T(A)=A+A^T$ - all matrices in the form $\begin{bmatrix}0 & b\\ -b & 0\end{bmatrix}$
			\end{itemize}
		\subsection{Let $T: P_2\rightarrow P_3$ be the linear transformaiton defined by $T(p(x))=xp(x)$. which are kernels to $T$}
			\begin{itemize}
				\item $x^2$ false
				\item $0$ true
				\item $x+1$ false
				\item $-x$ false
			\end{itemize}
		\subsection{Consider the basis $S=\{v_1,v_2\}$ for $R^2$ where $v_1=(1,1)$ and $v_2=(1,0)$ and let $T$ be a linear transformation, find $T(5,-3)$}
			\begin{align*}
				\begin{bmatrix}x_1&x_3\\x_2&x_4\end{bmatrix}\cdot \begin{bmatrix}1\\1\end{bmatrix}=\begin{bmatrix}1\\-2\end{bmatrix}\\
				\begin{bmatrix}x_1&x_3\\x_2&x_4\end{bmatrix}\cdot \begin{bmatrix}1\\0\end{bmatrix}=\begin{bmatrix}-4\\1\end{bmatrix}\\
				x_1+x_3=1\\
				x_2+x_4=-2\\
				x_1=-4\\
				x_2=1\\
				x_3=1-(-4)=5\\
				x_4=-2-1=-3\\
				\begin{bmatrix}-4&5\\1&-3\end{bmatrix}\cdot \begin{bmatrix}5\\-3\end{bmatrix}=\begin{bmatrix}-35\\14\end{bmatrix}
			\end{align*}
		\subsection{Are the following operator one-to-one}
			\begin{itemize}
				\item Orthogonal projection onto x axis in $R^2$ - false
				\item The reflection abouth the y axis in $R^2$ - true
				\item The reflection about the line $y=x$ in $R^2$ - true
			\end{itemize}
		\subsection{Are the following transformation one-to-one, determine using kernel}
			\begin{itemize}
				\item $T: R^2\rightarrow R^2$ where $T(x,y)=(y,x)$ - only kernel is $\{(0,0)\}$ therefore it is one-to-one
				\item $T: R^3\rightarrow R^2$ where $T(x,y,z)=(x+y+z,x-y-z)$ - only kernel is $\{(0,0)\}$ therefore it is one-to-one
			\end{itemize} 
		\subsection{Is multiplication by the matrix one-to-one?}
			\begin{align*}
				\begin{bmatrix}1&2\\2&-4\\-3&6\end{bmatrix}\cdot \begin{bmatrix}x_1\\x_2\end{bmatrix}=\begin{bmatrix}0\\0\\0\end{bmatrix}\\
				1x_1-2x_2=0\\
				2x_1-4x_2=0\\
				-3x_1+6x_2=0\\
				x_1=2x_2\\
			\end{align*}
			It can be seen as long that $x_1=2x_2$ the result will be zero therefore the kernel is not $\{0\}$ therefore making it not one-to-one
		\subsection{Dertermine if the transformation is one-to-one}
			\begin{itemize}
				\item $T:V\rightarrow W$; nullity$(T)=0$ - T is one-to-one
				\item $T:V\rightarrow W$ rank$(T)=dim(V)$ - T is one-to-one
				\item $T:V\rightarrow W$ $dim(W)<dim(V)$ - T is not one-to-one
			\end{itemize}		
	\section{Week 40}
		\subsection{Determine if the transformation is isomorphic}
			$$c_0+c_1x\rightarrow (c_0-c_1,c_1)$$
			From $P_1$ to $R^2$\\
			$ker(T):\{c_0=0,c_1=0\}$ therefore making it one-to-one, and onto since every vector is obtainable, making it isomorphic
		\subsection{Determine if the transformation is isomorphic}
			$$a+bx+cx^2+dx^3\rightarrow \begin{bmatrix}a&b\\c&d\end{bmatrix}$$
			from $P_3$ to $M_{22}$.\\
			Isomorphic
		\subsection{Find isomorphism between the spaces}
			\subsubsection{All $3\times 3$ symmetric matrices and $R^6$}
				$\begin{bmatrix}x_1&x_2&x_3\\x_2&x_4&x_5\\x_3&x_5&x_6\end{bmatrix}\rightarrow (x_1,x_2,x_3,x_4,x_5,x_6)$
			\subsubsection{All $2\times 2$ matrices and $R^4$ in two ways}
				$\begin{bmatrix}x_1&x_2\\x_3&x_4\end{bmatrix}$\\
				$(x_1,x_2,x_3,x_4)$
				$(x_2,x_3,x_4,x_1)$
		\subsection{Is the transformation isomorphic?}
			\begin{itemize}
				\item $\begin{bmatrix}0&1&-1\\1&0&2\\-1&1&0\end{bmatrix}$ the determinant is -3 therefore it is linear independent and isomorphic
				\item $\begin{bmatrix}1&-1&0\\0&0&2\\-1&1&0\end{bmatrix}$ the determinant is 0 therefore linear dependent and not isomorphic
			\end{itemize}
		\subsection{Let $T:P_2\rightarrow P_1$ be the linear transformation $T(a_1+a_1x+a_2x^2)=(a_0+a_1)-(2a_1+3a_2)x$}
			\subsubsection{Find the matrix for the linear transformation $T$ relatie to the standard bases $B=\{1,x,x^2\}$ and $B'=\{1,x\}$ for $P_2$ and $P_1$}
				\begin{align*}
					b_1=\begin{bmatrix}1\\0\\0\end{bmatrix}\\
					b_2=\begin{bmatrix}0\\1\\0\end{bmatrix}\\
					b_3=\begin{bmatrix}0\\0\\1\end{bmatrix}\\[3mm]
					T(b_1)=1=\begin{bmatrix}1\\0\\0\end{bmatrix}\\
					T(b_2)=1-2x=\begin{bmatrix}1\\-2\\0\end{bmatrix}\\
					T(b_3)=-3x=\begin{bmatrix}0\\-3\\0\end{bmatrix}\\[3mm]
					[T(b_1)]_{B'}=b'_1=\begin{bmatrix}1\\0\end{bmatrix}\\
					[T(b_2)]_{B'}=b'_1-2b'_2=\begin{bmatrix}1\\-2\end{bmatrix}\\
					[T(b_3)]_{B'}=-3b'_2=\begin{bmatrix}0\\-3\end{bmatrix}\\[3mm]
					[T]_{B',B}=\begin{bmatrix}1&1&0\\0&-2&-3\end{bmatrix}
				\end{align*}
			\subsubsection{Validate the matrix $[T]_{B',B}$ satisfies for every vector $x=c_0+c_1x+c_2x^2$ in $P_2$}
				$$\begin{bmatrix}1&1&0\\0&-2&-3\end{bmatrix}\begin{bmatrix}c_0,c_1,c_2\end{bmatrix}=\begin{bmatrix}c_0+c_2x\\-2c_2-3c_3\end{bmatrix}$$
		\subsection{Let $T:R^2\rightarrow R^2$ be the linear operator defined by $T(\begin{bmatrix}x_1\\x_2\end{bmatrix})=\begin{bmatrix}x_1-x_2\\x_1+x_2\end{bmatrix}$ and let $B=\{\begin{bmatrix}1\\1\end{bmatrix},\begin{bmatrix}-1\\0\end{bmatrix}\}$ be the basis}
			\subsubsection{Find $[T]_B$}
				\begin{align*}
					T(\begin{bmatrix}1\\1\end{bmatrix})=\begin{bmatrix}0\\2\end{bmatrix}\\
					T(\begin{bmatrix}-1\\0\end{bmatrix}=\begin{bmatrix}-1\\-1\end{bmatrix}\\
					[T]_B=\begin{bmatrix}0&-1\\2&-1\end{bmatrix}
				\end{align*}
			\subsubsection{Verify $[T]_B$}
				\begin{align*}
					[T]_B[x]_B=[T(x)]_B\\
					\begin{bmatrix}0&-1\\2&-1\end{bmatrix}\begin{bmatrix}1\\2\end{bmatrix}=\begin{bmatrix}-2\\0\end{bmatrix}\\
					T(\begin{bmatrix}1\\2\end{bmatrix})=\begin{bmatrix}-1\\3\end{bmatrix}
				\end{align*}
		\subsection{Let $T:R^2\rightarrow R^3$ be $T(\begin{bmatrix}x_1\\x_2\end{bmatrix})=\begin{bmatrix}x_1+2x_2\\-x_1\\0\end{bmatrix}$}
			\subsubsection{Find the matrix $[T]_{B',B}$ relative to bases $B'=\{v_1,v_2,v_3\}$ and $B=\{u_1,u_2\}$}
				\begin{align*}
					u_1=\begin{bmatrix}1\\3\end{bmatrix}\\
					u_2=\begin{bmatrix}-2\\4\end{bmatrix}\\
					v_1=\begin{bmatrix}1\\1\\1\end{bmatrix}\\
					v_2=\begin{bmatrix}2\\2\\0\end{bmatrix}\\
					v_3=\begin{bmatrix}3\\0\\0\end{bmatrix}\\
					T(u_1)=\begin{bmatrix}7\\-1\\0\end{bmatrix}\\
					T(u_2)=\begin{bmatrix}6\\2\\0\end{bmatrix}\\
					[T(u_1)]_B=\begin{bmatrix}0\\-0.5\\\frac{8}{3}\end{bmatrix}\\
					[T(u_2)]_B=\begin{bmatrix}0\\1\\\frac{4}{3}\end{bmatrix}\\
					[T]_{B',B}=\begin{bmatrix}0&0\\-0.5&1\\\frac{8}{3}&\frac{4}{3}\end{bmatrix}
				\end{align*}
		\subsection{Let $v_1=\begin{bmatrix}1\\3\end{bmatrix}$ and $v_2=\begin{bmatrix}-1\\4\end{bmatrix}$, and let $A=\begin{bmatrix}1&3\\-2&5\end{bmatrix}$ be the matrix for $T:R^2\rightarrow R^2$ relative to the basis $B=\{v_1,v_2\}$}
			\subsubsection{Find $[T(v_1)]_B$ and $[T(v_2)]_B$}
				\begin{align*}
					[T(v_1)]_B=\begin{bmatrix}1\\-2\end{bmatrix}
					[T(v_2)]_B=\begin{bmatrix}3\\-5\end{bmatrix}\}
				\end{align*}
			\subsubsection{Find $T(v_1)$ and $T(v_2)$}
					
		\subsection{Let $D:P_2\rightarrow P_2$ be the differentiation operator $D(P)=p'(x)$}
			\subsubsection{Find the matrix for $D$ relative to the basis $B=\{p_1,p_2,p_3\}$ for $P_2$ in which $p_1=1,p_2=x,p_3=2-3x+8x^2$}
				$$\begin{bmatrix} 0 & 1 & 3\\0&0&16\\0&0&0\end{bmatrix}$$
			\subsubsection{Use the matrix to compute $D(6-6x+23x^2)$}
				$$\begin{bmatrix} 0 & 1 & 3\\0&0&16\\0&0&0\end{bmatrix}\begin{bmatrix}6\\-6\\23\end{bmatrix}=\begin{bmatrix}63\\368\\0\end{bmatrix}$$
				$$63+368x$$
		\subsection{Let $T:R^2\rightarrow R^2$ be a linear operator and let $B$ and $B'$ be bases for $R^2$ for which $[T]_B=\begin{bmatrix}2&0\\1&1\end{bmatrix}$ and $P_{B\rightarrow B'}=\begin{bmatrix}3&2\\1&1\end{bmatrix}$. Find the matrix for $T$ relative to the basis $B'$}
			\begin{align*}
				P^{-1}=P_{B'\rightarrow B} = \begin{bmatrix}1&-2\\-1&3\end{bmatrix}\\
				T_{B'}=P_{B\rightarrow B'}T_BP_{B'\rightarrow B}\\
				\begin{bmatrix}3&2\\1&1\end{bmatrix}\begin{bmatrix}2&0\\1&1\end{bmatrix}=\begin{bmatrix}8&2\\3&1\end{bmatrix}
				\begin{bmatrix}8&2\\3&1\end{bmatrix} \begin{bmatrix}1&-2\\-1&3\end{bmatrix}=\begin{bmatrix}6&-10\\2&-3\end{bmatrix}
			\end{align*}
		\subsection{$T:P_1\rightarrow P_1$ is defined by $T(a_0+a_1x)=-a_0+(a_0+a_1)x$ $B$ is the standard basis for $P_1$ and $B'=\{q_1,q_2\}$ where $q_1=x+1, q_2=x-1$. Find T relative to the basis $B$}
			\begin{align*}
				T_B=\begin{bmatrix}-1&0\\1&1\end{bmatrix}\\
				B'=\begin{bmatrix}1&-1\\1 & 1\end{bmatrix}\\
				B'^{-1}=\begin{bmatrix}0.5&0.5\\-0.5&0.5\end{bmatrix}\\
				B'^{-1}T_BB'=\begin{bmatrix}0.5&0.5\\1.5&-0.5\end{bmatrix}
			\end{align*}
		\subsection{Check that $x$ is an eigenvector of $A$ and find the eigenvalue}
			\begin{align*}
				A=\begin{bmatrix}1&2\\3&2\end{bmatrix}\\
				x=\begin{bmatrix}1\\-1\end{bmatrix}\\
				Ax=\begin{bmatrix}-1\\1\end{bmatrix}\\
				\lambda = -1
			\end{align*}
		\subsection{Check that $x$ is an eigenvector of $A$ and find the eigenvalue}
			\begin{align*}
				A=\begin{bmatrix}4 & 0 & 1\\ 2 & 3 &2 \\ 1 & 0 & 4\end{bmatrix}\\
				x=\begin{bmatrix}1\\2\\1\end{bmatrix}\\
				Ax=\begin{bmatrix}5\\ 10 \\ 5\end{bmatrix}\\
				\lambda = 5
			\end{align*}
		\subsection{Find the characteristic equation, the eigenvalues, and bases for the eigenspaces of the matrix}
			\subsubsection{$\begin{bmatrix}1&4\\2&3\end{bmatrix}$}
				\begin{align*}
					det(\begin{bmatrix}\lambda & 0\\0&\lambda \end{bmatrix}-\begin{bmatrix}1&4\\2&3\end{bmatrix})=0\\
					det(\begin{bmatrix}\lambda-1 & -4\\ -2 & \lambda -3\end{bmatrix})=0\\
					(\lambda-1)(\lambda-3)-(-4\cdot -2)=0\\
					\lambda^2-\lambda-3\lambda+3-8=0\\
					\lambda^2-4\lambda-5=0\\
					\lambda=(5,-1)\\[4mm]
					\begin{bmatrix}\lambda-1 & -4\\ -2 & \lambda -3\end{bmatrix}\begin{bmatrix}x_1\\x_2\end{bmatrix}=\begin{bmatrix}0\\0\\0\end{bmatrix}\\
					\begin{bmatrix}4 & -4\\ -2 & 2 \end{bmatrix}\begin{bmatrix}x_1\\x_2\end{bmatrix}=\begin{bmatrix}0\\0\end{bmatrix}\\
					x_1-x_2=0\\
					\begin{bmatrix}1\\-1\end{bmatrix}\\
					\begin{bmatrix}\lambda-1 & -4\\ -2 & \lambda -3\end{bmatrix}\begin{bmatrix}x_1\\x_2\end{bmatrix}=\begin{bmatrix}0\\0\\0\end{bmatrix}\\
					\begin{bmatrix}-2 & -4\\ -2 & -4 \end{bmatrix}\begin{bmatrix}x_1\\x_2\end{bmatrix}=\begin{bmatrix}0\\0\end{bmatrix}\\
					x_1+2x_2=0\\
					\begin{bmatrix}-2\\1\end{bmatrix}
				\end{align*}
			\subsubsection{$\begin{bmatrix}-2&-7\\1&2\end{bmatrix}$}
				\begin{align*}
					det(\begin{bmatrix}\lambda & 0\\0&\lambda \end{bmatrix}-\begin{bmatrix}-2&-7\\1&2\end{bmatrix})=0\\
					det(\begin{bmatrix}\lambda+2 & 7\\ -1 & \lambda -2\end{bmatrix})=0\\
					(\lambda-2)(\lambda-2)-(7\cdot -1)=0\\
					\lambda^2+3=0
				\end{align*}
				No real solutions
			\subsubsection{$\begin{bmatrix}4&0&1\\-2&1&0\\-2&0&1\end{bmatrix}$}
				\begin{align*}
					det(\begin{bmatrix}\lambda & 0 & 0\\0&\lambda & 0\\ 0 & 0 & \lambda\end{bmatrix}-\begin{bmatrix}4&0&1\\-2&1&0\\-2&0&1\end{bmatrix})=0\\
					det(\begin{bmatrix}\lambda-4&0&-1\\2&\lambda-1&0\\2&0&\lambda-1\end{bmatrix})=0\\
					x^3-6x^2+11x-6=0\\
					x=(1,2,3)\\
					\hline\\
					x=1\\
					\begin{bmatrix}\lambda-4&0&-1\\2&\lambda-1&0\\2&0&\lambda-1\end{bmatrix}\\
					\begin{bmatrix}-3&0&-1\\2&0&0\\2&0&0\end{bmatrix}\begin{bmatrix}x_1\\x_2\\x_3\end{bmatrix}=\begin{bmatrix}0\\0\\0\end{bmatrix}\\
					x_1=0\land x_3=0\\
					\begin{bmatrix}0\\1\\0\end{bmatrix}\\
					\hline\\
					x=2\\
					\begin{bmatrix}\lambda-4&0&-1\\2&\lambda-1&0\\2&0&\lambda-1\end{bmatrix}\\
					\begin{bmatrix}-2&0&-1\\2&1&0\\2&0&1\end{bmatrix}\begin{bmatrix}x_1\\x_2\\x_3\end{bmatrix}=\begin{bmatrix}0\\0\\0\end{bmatrix}\\
					x_1-0.5x_3=0\land x_2-x_3=0\\
					\begin{bmatrix}0.5\\1\\1\end{bmatrix}\\
					\hline\\
				\end{align*}
				\begin{align*}
					x=3\\
					\begin{bmatrix}\lambda-4&0&-1\\2&\lambda-1&0\\2&0&\lambda-1\end{bmatrix}\\
					\begin{bmatrix}-1&0&-1\\2&2&0\\2&0&2\end{bmatrix}\begin{bmatrix}x_1\\x_2\\x_3\end{bmatrix}=\begin{bmatrix}0\\0\\0\end{bmatrix}\\
					x_1+x_3=0\\
					x_2-x_3=0\\
					\begin{bmatrix}1\\1\\-1\end{bmatrix}
				\end{align*}
		\subsection{Find the characteristic equation by inspection}
			$$\begin{bmatrix}3&0&0\\-2&7&0\\4&8&1\end{bmatrix}$$
			Lower triangle therefore
			$$(x-3)(x-7)(x-1)=0$$
		\subsection{Find eigenvalues and basis, for $T(x,y)=(x+4y,2x+3y)$}
			\begin{align*}
				\begin{bmatrix}x&4y\\2x&3y\end{bmatrix}\\
				det(\begin{bmatrix}\lambda&0\\0&\lambda\end{bmatrix}-\begin{bmatrix}1&4\\2&3\end{bmatrix})=0\\
				det(\begin{bmatrix}\lambda-1&-4\\-2&\lambda-3\end{bmatrix})=0\\
				x^2-4x-5=0\\
				x=(5,-1)\\
				\hline
				x=5\\
				\begin{bmatrix}4&-4\\-2&2\end{bmatrix}\begin{bmatrix}x_1\\x_2\end{bmatrix}=\begin{bmatrix}0\\0\end{bmatrix}\\
				x_1-x_2=0\\
				\begin{bmatrix}1\\1\end{bmatrix}\\
				\hline
				x=-1\\
				\begin{bmatrix}-2&-4\\-2&-4\end{bmatrix}\begin{bmatrix}x_1\\x_2\end{bmatrix}=\begin{bmatrix}0\\0\end{bmatrix}\\
				x_1+2x_2=0\\
				\begin{bmatrix}-2\\1\end{bmatrix}
			\end{align*}
		\subsection{Show that $A$ and $B$ are not similar matrices}
			\subsubsection{$A=\begin{bmatrix}1&1\\3&2\end{bmatrix}, B=\begin{bmatrix}1 & 0 \\ 3 & -2\end{bmatrix}$}
				$det(A)=-1$\\
				$det(B)=-2$\\
				Therefore since they do not equal they can not be similar
		\subsection{Find the matrix $P$ that diagonalizes $\begin{bmatrix}2&0&-2\\0&3&0\\0&0&3\end{bmatrix}$}
			\begin{align*}
				det(\begin{bmatrix}\lambda-2&0&2\\0&\lambda-3&0\\0&0&\lambda-3\end{bmatrix})=0\\
				(x-2)(x-3)^2=0\\
				x=(2,3)\\
				\hline\\
				x=2\\
				\begin{bmatrix}0&0&2\\0&-1&0\\0&0&-1\end{bmatrix}\begin{bmatrix}x_1\\x_2\\x_3\end{bmatrix}=\begin{bmatrix}0\\0\\0\end{bmatrix}\\
				x_2=0\\
				x_3=0\\
				\begin{bmatrix}1\\0\\0\end{bmatrix}\\
				\hline\\
				x=3\\
				\begin{bmatrix}1&0&2\\0&0&0\\0&0&0\end{bmatrix}\begin{bmatrix}x_1\\x_2\\x_3\end{bmatrix}=\begin{bmatrix}0\\0\\0\end{bmatrix}\\
				x_1+2x_3=0\\
				\begin{bmatrix}-2\\0\\1\end{bmatrix}\\
				\begin{bmatrix}0\\1\\0\end{bmatrix}\\
				\hline\\
				P=\begin{bmatrix}1&0&-2\\0&1&0\\0&0&1\end{bmatrix}\\
				P^{-1}=\begin{bmatrix}1&0&2\\0&1&0\\0&0&1\end{bmatrix}\\
				P^{-1}AP=\begin{bmatrix}2&0&0\\0&3&0\\0&0&3\end{bmatrix}
			\end{align*}
		\subsection{Find the geometric and algebraic multiplicity of each eigenvalue of $\begin{bmatrix}-1&4&-2\\-3&4&0\\-3&1&3\end{bmatrix}$, and if it diagonalizable}
			\begin{align*}
				det(\begin{bmatrix}\lambda+1&-4&2\\3&\lambda-4&0\\3&-1&\lambda-3\end{bmatrix})=0\\
				x^3-6x^2+11x-6=0\\
				x=(1,2,3)\\
				\hline\\
				x=1\\
				alg: 1 \\
				\begin{bmatrix}2&-4&2\\3&-3&0\\3&-1&-2\end{bmatrix}\begin{bmatrix}x_1\\x_2\\x_3\end{bmatrix}=\begin{bmatrix}0\\0\\0\end{bmatrix}\\
				x_1-x_3=0\\
				x_2-x_3=0\\
				\begin{bmatrix}1\\1\\1\end{bmatrix}\\
				geo: 1 \text{ since the whole space can be made with 1 vector}\\
				\hline\\
				x=2\\
				alg: 1 \\
				\begin{bmatrix}3&-4&2\\3&-2&0\\3&-1&-1\end{bmatrix}\begin{bmatrix}x_1\\x_2\\x_3\end{bmatrix}=\begin{bmatrix}0\\0\\0\end{bmatrix}\\
				x_1-\frac{2}{3}x_3=0\\
				x_2-x_3=0\\
				\begin{bmatrix}\frac{2}{3}\\1\\1\end{bmatrix}\\
				geo: 1 \text{ since the whole space can be made with 1 vector}\\
				\hline
			\end{align*}
			\begin{align*}
				x=3\\
				alg: 1\\
				\begin{bmatrix}4&-4&2\\3&-1&0\\3&-1&0\end{bmatrix}\begin{bmatrix}x_1\\x_2\\x_3\end{bmatrix}=\begin{bmatrix}0\\0\\0\end{bmatrix}\\
				x_1-0.25x_3=0\\
				x_2-0.75x_3=0\\
				\begin{bmatrix}0.25\\0.75\\1\end{bmatrix}\\
				geo: 1 \text{ since the whole space can be made with 1 vector}				 
			\end{align*}
		\subsection{Compute $\begin{bmatrix}0&3\\2&-1\end{bmatrix}^{10}$}
			\begin{align*}
				det(\begin{bmatrix}\lambda&-3\\-2&\lambda+1\end{bmatrix})=0\\
				x^2+x-6=0\\
				x=(2,-3)\\
				\hline\\
				x=2\\
				\begin{bmatrix}2&-3\\-2&3\end{bmatrix}\begin{bmatrix}x_1\\x_2\end{bmatrix}=\begin{bmatrix}0\\0\end{bmatrix}\\
				x_1-1.5x_2=0\\
				\begin{bmatrix}1.5\\1\end{bmatrix}\\
				\hline\\
				x=-3\\
				\begin{bmatrix}-3&-3\\-2&-2\end{bmatrix}\begin{bmatrix}x_1\\x_2\end{bmatrix}=\begin{bmatrix}0\\0\end{bmatrix}\\
				x_1+x_2=0\\
				\begin{bmatrix}1\\-1\end{bmatrix}\\
				\hline\\
				P=\begin{bmatrix}1.5&1\\1&-1\end{bmatrix}\\
				P^{-1}=\begin{bmatrix}0.4&0.4\\0.4&-0.6\end{bmatrix}\\
				D=P^{-1}AP=\begin{bmatrix}2&0&0\\0&3&0\\0&0&3\end{bmatrix}\\
				D=\begin{bmatrix}2&0\\0&-3\end{bmatrix}\\
				A^{10}=PD^{10}p^{-1}\\
				A^{10}=\begin{bmatrix}24234 & -34815\\-23210 & 35839\end{bmatrix}
			\end{align*}
	\section{Week 41}
		\subsection{Compute the standard inner product of $U=\begin{bmatrix}3 & -2\\ 4 & 8\end{bmatrix}$, $V=\begin{bmatrix}-1&3\\1&1\end{bmatrix}$}
			\begin{align*}
				<U,V>\\
				3\cdot (-1)+(-2)\cdot 3+4\cdot 1 + 8\cdot 1\\
				<U,V>=-3-6+4+8=3
			\end{align*}
		\subsection{Compute the standard inner product of $p=-2+x+3x^2$, $q=4-7x^2$}
			\begin{align*}
				<p,q>\\
				-2\cdot 4+1\cdot 0+3\cdot (-7)\\
				<p,q>=-8+0-21=-29
			\end{align*}
		\subsection{Determine whether the vectors are orthogonal with repsect to the euclidean inner product}
			\begin{itemize}
				\item $u=(-1,3,2),v=(4,2,-1)$ - $<u,v>= -4 + 6 -2 = 0$
				\item $u=(-2,-2,-2),v=(1,1,1)$ - $<u,v> = -2-2-2=-4$
				\item $u=(a,b),v=(-b,a)$ - $<u,v>=-ab+ab=0$
			\end{itemize}
			Only the last and the first vectors are orthogonal relative given the inner product is 0
		\subsection{Determine if the vectors are orthogonal and/or orthonormal respect to the inner product}
			\begin{itemize}
				\item $(0,1)$, $(2,0)$ - $<>=0$
				\item $(-\frac{1}{\sqrt{2}},\frac{1}{\sqrt{2}}),  (\frac{1}{\sqrt{2}},\frac{1}{\sqrt{2}})$  - $<>=0$
				\item $(-\frac{1}{\sqrt{2}},-\frac{1}{\sqrt{2}}),-\frac{1}{\sqrt{2}},-\frac{1}{\sqrt{2}}))$ - $<>=-1$
			\end{itemize}
			The two first are orthogonal relative and the third has an orthonormal basis
		\subsection{Transform the basis $\{(1,1,1),(-1,1,0),(1,2,1)\}$ into an orthonormal basis}
			\begin{align*}
				u_1=(1,1,1)\\
				u_2=(-1,1,0)-\frac{<(-1,1,0),(1,1,1)>}{|(1,1,1)|^2}(1,1,1)\\
				u_2=(-1,1,0)-\frac{0}{\sqrt{3}^2}(1,1,1)\\
				u_2=(-1,1,0)-(0,0,0)=(-1,1,0)\\
				u_3=(1,2,1)-\frac{<(1,2,1),(1,1,1)>}{|(1,1,1)|^2}(1,1,1)-\frac{<(1,2,1),(-1,1,0)>}{|(-1,1,0)|^2}(-1,1,0)\\
				u_3=(1,2,1)-\frac{4}{\sqrt{3}^2}(1,1,1)-\frac{1}{\sqrt{2}^2}(-1,1,0)\\
				u_3=(1,2,1)-(\frac{4}{3},\frac{4}{3},\frac{4}{3})-(-0.5,0.5,0)\\
				u_3=(\frac{1}{6},\frac{1}{6},-\frac{1}{3})\\
				u_1=\frac{1}{\sqrt{3}}(1,1,1)\\
				u_2=\frac{1}{\sqrt{2}}(-1,1,0)\\
				u_3=\frac{\sqrt{6}}{6}(\frac{1}{6},\frac{1}{6},-\frac{1}{3})
			\end{align*}
		\subsection{Determine if the matrix is orthogonal, and if so find the inverse}
			\begin{itemize}
				\item $a= \begin{bmatrix}1&0\\0&-1\end{bmatrix}$ - $a^T=a^{-1}=\begin{bmatrix}1&0\\0&-1\end{bmatrix}$
				\item $a=\begin{bmatrix}\frac{1}{\sqrt{2}} & -\frac{1}{\sqrt{2}}\\\frac{1}{\sqrt{2}}&\frac{1}{\sqrt{2}}\end{bmatrix}$ -  $a^T=a^{-1}=\begin{bmatrix}\frac{1}{\sqrt{2}} & \frac{1}{\sqrt{2}}\\-\frac{1}{\sqrt{2}}&\frac{1}{\sqrt{2}}\end{bmatrix}$
			\end{itemize}
		\subsection{For the matrix $A=\begin{bmatrix}\frac{3}{2}&-\frac{1}{2}\\-\frac{1}{2}&\frac{3}{2}\end{bmatrix}$}
			\subsubsection{Verify that $A$ is orthogonally diagonalisable}
				It can be seen that $A$ is symmetric therefore it is orthogonally diagonalizable
			\subsubsection{Find the orthogonal matrix $P$ such that $P^{-1}AP$ is diagonal}
				\begin{align*}
					det(\begin{bmatrix}\lambda-\frac{3}{2} & \frac{1}{2}\\\frac{1}{2}&\lambda-\frac{3}{2}\end{bmatrix})=0\\
					x^2-3x+2=0\\
					x=(1,2)\\
					\hline\\
					x=1\\
					(\begin{bmatrix}-\frac{1}{2} & \frac{1}{2}\\\frac{1}{2}&-\frac{1}{2}\end{bmatrix}\begin{bmatrix}x_1\\x_2\end{bmatrix}=\begin{bmatrix}0\\0\end{bmatrix}\\
					x_1-x_2=0
					\begin{bmatrix}1\\1\end{bmatrix}\\
					\hline\\
					x=2\\
					\begin{bmatrix}\frac{1}{2} & \frac{1}{2}\\\frac{1}{2}&\frac{1}{2}\end{bmatrix}\begin{bmatrix}x_1\\x_2\end{bmatrix}=\begin{bmatrix}0\\0\end{bmatrix}\\
					x_1+x_2=0\\
					\begin{bmatrix}1\\-1\end{bmatrix}\\
					\hline\\
					P=\begin{bmatrix}1&1\\1&-1\end{bmatrix}\\
					P^{-1}=\begin{bmatrix}0.5&0.5\\0.5&-0.5\end{bmatrix}\\
					D=P^TAP=\begin{bmatrix}1&0\\0&2\end{bmatrix}
				\end{align*}
			\subsubsection{Calculate $A^{10}$}
				$$P^{-1}D^{10}P=\begin{bmatrix}512.5&-511.5\\-511.5&512.5\end{bmatrix}$$
		\subsection{Find the characteristic equation, and the dimensions of the eigenspaces by inspections}
			\subsubsection{$\begin{bmatrix}1&2\\2&4\end{bmatrix}$}
				\begin{align*}
					det(\begin{bmatrix}\lambda-1&-2\\-2&\lambda-4\end{bmatrix})=0\\
					x^2-5x=0\\
					x=(0,5)\\
					dim(0)=1\\
					dim(5)=1
				\end{align*}
			\subsubsection{$\begin{bmatrix}\lambda-1&4&-2\\4&\lambda-1&2\\-2&2&\lambda+2\end{bmatrix}$}
				\begin{align*}
					det\begin{bmatrix}\lambda-1&4&-2\\4&\lambda-1&2\\-2&2&\lambda+2\end{bmatrix})=0\\
					x=(6,-3,-3)\\
					dim(6)=3\\
					dim(-3)=2
				\end{align*}
\end{document}
