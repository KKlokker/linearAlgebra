\documentclass[12pt, a4paper]{article}
\usepackage{caption}
\usepackage{graphicx}
\usepackage{hyperref}
\hypersetup{
    colorlinks,
    citecolor=black,
    filecolor=black,
    linkcolor=black,
    urlcolor=black
}
\usepackage{tikz-network}
\usepackage{amsmath, amsfonts, amssymb, amsthm}
\usepackage{algpseudocode}
\usepackage{algorithm}
\title{Linear algebra\\ Exercises}
\date{2022}
\author{Kristoffer Klokker}

\usepackage{xcolor,listings}
\usepackage{textcomp}
\usepackage{color}
\usepackage{listings}
\definecolor{codegreen}{rgb}{0,0.6,0}
\definecolor{codegray}{rgb}{0.5,0.5,0.5}
\definecolor{codepurple}{HTML}{C42043}
\definecolor{backcolour}{HTML}{F2F2F2}
\definecolor{bookColor}{cmyk}{0,0,0,0.90}  
\color{bookColor}

\lstset{upquote=true}

\lstdefinestyle{mystyle}{
    backgroundcolor=\color{backcolour},   
    commentstyle=\color{codegreen},
    keywordstyle=\color{codepurple},
    numberstyle=\numberstyle,
    stringstyle=\color{codepurple},
    basicstyle=\footnotesize\ttfamily,
    breakatwhitespace=false,
    breaklines=true,
    captionpos=b,
    keepspaces=true,
    numbers=left,
    numbersep=10pt,
    showspaces=false,
    showstringspaces=false,
    showtabs=false,
    tabsize=3,
}
\lstset{style=mystyle}
\usepackage{zref-base}

\makeatletter
\newcounter{mylstlisting}
\newcounter{mylstlines}
\lst@AddToHook{PreSet}{%
  \stepcounter{mylstlisting}%
  \ifnum\mylstlines=1\relax
    \lstset{numbers=none}
  \else
    \lstset{numbers=left}
  \fi
  \setcounter{mylstlines}{0}%
}
\lst@AddToHook{EveryPar}{%
  \stepcounter{mylstlines}%
}
\lst@AddToHook{ExitVars}{%
  \begingroup
    \zref@wrapper@immediate{%
      \zref@setcurrent{default}{\the\value{mylstlines}}%
      \zref@labelbyprops{mylstlines\the\value{mylstlisting}}{default}%
    }%
  \endgroup
}

% \mylstlines print number of lines inside listing caption
\newcommand*{\mylstlines}{%
  \zref@extractdefault{mylstlines\the\value{mylstlisting}}{default}{0}%
}
\makeatother


\newcommand\numberstyle[1]{%
    \footnotesize
    \color{codegray}%
    \ttfamily
    \ifnum#1<10 0\fi#1 |%
}


\begin{document}
	\maketitle
	\clearpage
	\tableofcontents
	\clearpage
	\section{Week 36}
		\subsection{Which equation are linear in $x_1$, $x_2$ and $x_3$}
			\subsubsection{$x_1+5x_2-\sqrt{2}x^3=1$}
				This is a linear equation
			\subsubsection{$x_1=-7x_2+3x_3$}
				This is a linear equation
			\subsubsection{$x_1^{3/5}-2x_2+x_3=4$}
				This is not a linear equation with $x_1$ having a power
		\subsection{Convert from matrix form to equation form}
			$\begin{bmatrix}
				2 & 0 & 0 \\
				3 & -4 & 0\\
				0 & 1 & 1 
			\end{bmatrix}$\\
			$2x_1=0$\\
			$3x_1-4x_2=0$\\
			$x_2=1$
		\subsection{Convert from equations to matrix}
			$-6x_1-x_2+3x_3=4$\\
			$5x_2-x_3=1$\\
			$\begin{bmatrix}
				-6 & -1 & 3 & 4 \\
				0 & 5 & -1 & 1
			\end{bmatrix}$
		\subsection{Determine if the solution hold in the following system}
			$(5,8,1)$\\
			$x+2y-2z=3$\\
			$3x-y+z=1$\\
			$-x+5y-5z=5$\\[4mm]
			$5+2(8)-2(1)=3$\\
			$19=3$\\
			By the first equation the solution does not hold
		\subsection{Determine if the following matrices are in echoleon form or reduced echelon form}
			$\begin{bmatrix}
				1 & 0 & 0 \\
				0 & 1 & 0\\
				0 & 0 & 1 
			\end{bmatrix}$\\
			Reduced echelon form\\
			$\begin{bmatrix}
				1 & -3 & 4 & 7 \\
				0 & 1 & 2 & 2\\
				0 & 0 & 1  & 5
			\end{bmatrix}$\\
			Echelon form
	\section{Week 38}
		\subsection{Show that the determinant og the matix is 0}
			$\begin{bmatrix}
				-2 & 8 & 1 & 4 \\
				3 & 2 & 5 & 1\\
				1 & 10 & 6  & 5\\
				4 & -6 & 4  & -3
			\end{bmatrix}$\\[3mm]
			Column 2 and 4 are proportional to eachother therefore making the det=0
		\subsection{Is the following matrix invertible}
			$\begin{bmatrix}
				2 & 0 & 3 \\
				0 & 3 & 2 \\
				-2 & 0 & -4
			\end{bmatrix}$\\[3mm]
			The determinant is -6 and therefore not zero and therefore invertible
		\subsection{Find the standard matrix for the transformation defined by the equations}
			\begin{align*}
				w_1=7x_1+2x_2-8x_3\\
				w_2=-x_2+5x_3\\
				w_3=4x_1+7x_2-x_3
			\end{align*}
			$\begin{bmatrix}
				7 & 2 & -8 \\
				0 & -1 & 5 \\
				4 & 7 & -1
			\end{bmatrix}$\\[3mm]
		\subsection{Is a linear function a transformation of $R$}
			Yes the linear function can be a transform of the space $R$
		\subsection{The images of the standard basis vectors for $R^3$ are given for a linear trasnformation $T:R^3\rightarrow R^3$. Find the standard matrix for the transformation and find $T(x)$}
			\begin{align*}
					T(e_1)=
				\begin{bmatrix}
					1 \\
					3 \\
					0
				\end{bmatrix}\\[3mm]
					T(e_2)=
				\begin{bmatrix}
					0 \\
					0 \\
					1
				\end{bmatrix}\\[3mm]
					T(e_3)=
				\begin{bmatrix}
					4 \\
					-3 \\
					-1
				\end{bmatrix}\\[3mm]
					x=
				\begin{bmatrix}
					2 \\
					1 \\
					0
				\end{bmatrix}\\[3mm]
					A=
				\begin{bmatrix}
					1 & 0 & 4 \\
					3 & 0 & -3\\
					0 & 1 & -1
				\end{bmatrix}\\[3mm]
					T(x)=
				\begin{bmatrix}
					2 \\
					6 \\
					1
				\end{bmatrix}\\[3mm]
			\end{align*}
		\subsection{Find the standard matrix $A$ for the linear transformation $T:R^2\rightarrow R^2$ for which}
			$$T(\begin{bmatrix}
					1 \\
					1
				\end{bmatrix})=\begin{bmatrix}
					1 \\
					-2
				\end{bmatrix},T(\begin{bmatrix}
					2 \\
					3
				\end{bmatrix})=\begin{bmatrix}
					-2 \\
					5
				\end{bmatrix}$$
				
				
			\begin{align*}
				\begin{bmatrix}
					1 \\
					0
				\end{bmatrix}=c_1\begin{bmatrix}
					1 \\
					1
				\end{bmatrix}+c_2\begin{bmatrix}
					2 \\
					3
				\end{bmatrix}\\
				\begin{bmatrix}
					0 \\
					1
				\end{bmatrix}=k_1\begin{bmatrix}
					1 \\
					1
				\end{bmatrix}+k_2\begin{bmatrix}
					2 \\
					3
				\end{bmatrix}\\
				\text{c variable found by}\\
				\begin{bmatrix}
					1 & 2 & 1 \\
					1 & 3 & 0
				\end{bmatrix}\\
				\text{k variable found by}\\
				\begin{bmatrix}
					1 & 2 & 0 \\
					1 & 3 & 1
				\end{bmatrix}\\
				\text{Gauss jordian magic!}\\
				c_1=3, c_2=-1, k_1=-2,k_2=1\\
				T(
				\begin{bmatrix}
					1 \\
					0
				\end{bmatrix})=3T(
				\begin{bmatrix}
					1 \\
					1
				\end{bmatrix})+(-1)T(
				\begin{bmatrix}
					2 \\
					3
				\end{bmatrix})\\
				=\begin{bmatrix}
					3 \\
					-6
				\end{bmatrix}-
				\begin{bmatrix}
					-2 \\
					5
				\end{bmatrix}=\begin{bmatrix}
					5 \\
					-11
				\end{bmatrix}\\
				T(
				\begin{bmatrix}
					0 \\
					1
				\end{bmatrix})=-2T(
				\begin{bmatrix}
					1 \\
					1
				\end{bmatrix})+1T(
				\begin{bmatrix}
					2 \\
					3
				\end{bmatrix})\\
				=\begin{bmatrix}
					-2 \\
					4
				\end{bmatrix}+
				\begin{bmatrix}
					-2 \\
					5
				\end{bmatrix}=\begin{bmatrix}
					-4 \\
					9
				\end{bmatrix}\\[4mm]
				A=\begin{bmatrix}
				5&-4\\
				-11 & 9
					\end{bmatrix}
			\end{align*}
\end{document}
