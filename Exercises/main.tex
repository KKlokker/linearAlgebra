\documentclass[12pt, a4paper]{article}
\usepackage{caption}
\usepackage{graphicx}
\usepackage{hyperref}
\hypersetup{
    colorlinks,
    citecolor=black,
    filecolor=black,
    linkcolor=black,
    urlcolor=black
}
\usepackage{tikz-network}
\usepackage{amsmath, amsfonts, amssymb, amsthm}
\usepackage{algpseudocode}
\usepackage{algorithm}
\title{Linear algebra\\ Exercises}
\date{2022}
\author{Kristoffer Klokker}

\usepackage{xcolor,listings}
\usepackage{textcomp}
\usepackage{color}
\usepackage{listings}
\definecolor{codegreen}{rgb}{0,0.6,0}
\definecolor{codegray}{rgb}{0.5,0.5,0.5}
\definecolor{codepurple}{HTML}{C42043}
\definecolor{backcolour}{HTML}{F2F2F2}
\definecolor{bookColor}{cmyk}{0,0,0,0.90}  
\color{bookColor}

\lstset{upquote=true}

\lstdefinestyle{mystyle}{
    backgroundcolor=\color{backcolour},   
    commentstyle=\color{codegreen},
    keywordstyle=\color{codepurple},
    numberstyle=\numberstyle,
    stringstyle=\color{codepurple},
    basicstyle=\footnotesize\ttfamily,
    breakatwhitespace=false,
    breaklines=true,
    captionpos=b,
    keepspaces=true,
    numbers=left,
    numbersep=10pt,
    showspaces=false,
    showstringspaces=false,
    showtabs=false,
    tabsize=3,
}
\lstset{style=mystyle}
\usepackage{zref-base}

\makeatletter
\newcounter{mylstlisting}
\newcounter{mylstlines}
\lst@AddToHook{PreSet}{%
  \stepcounter{mylstlisting}%
  \ifnum\mylstlines=1\relax
    \lstset{numbers=none}
  \else
    \lstset{numbers=left}
  \fi
  \setcounter{mylstlines}{0}%
}
\lst@AddToHook{EveryPar}{%
  \stepcounter{mylstlines}%
}
\lst@AddToHook{ExitVars}{%
  \begingroup
    \zref@wrapper@immediate{%
      \zref@setcurrent{default}{\the\value{mylstlines}}%
      \zref@labelbyprops{mylstlines\the\value{mylstlisting}}{default}%
    }%
  \endgroup
}

% \mylstlines print number of lines inside listing caption
\newcommand*{\mylstlines}{%
  \zref@extractdefault{mylstlines\the\value{mylstlisting}}{default}{0}%
}
\makeatother


\newcommand\numberstyle[1]{%
    \footnotesize
    \color{codegray}%
    \ttfamily
    \ifnum#1<10 0\fi#1 |%
}


\begin{document}
	\maketitle
	\clearpage
	\tableofcontents
	\clearpage
	\section{Week 46}
		\subsection{Which equation are linear in $x_1$, $x_2$ and $x_3$}
			\subsubsection{$x_1+5x_2-\sqrt{2}x^3=1$}
				This is a linear equation
			\subsubsection{$x_1=-7x_2+3x_3$}
				This is a linear equation
			\subsubsection{$x_1^{3/5}-2x_2+x_3=4$}
				This is not a linear equation with $x_1$ having a power
		\subsection{Convert from matrix form to equation form}
			$\begin{bmatrix}
				2 & 0 & 0 \\
				3 & -4 & 0\\
				0 & 1 & 1 
			\end{bmatrix}$\\
			$2x_1=0$\\
			$3x_1-4x_2=0$\\
			$x_2=1$
		\subsection{Convert from equations to matrix}
			$-6x_1-x_2+3x_3=4$\\
			$5x_2-x_3=1$\\
			$\begin{bmatrix}
				-6 & -1 & 3 & 4 \\
				0 & 5 & -1 & 1
			\end{bmatrix}$
		\subsection{Determine if the solution hold in the following system}
			$(5,8,1)$\\
			$x+2y-2z=3$\\
			$3x-y+z=1$\\
			$-x+5y-5z=5$\\[4mm]
			$5+2(8)-2(1)=3$\\
			$19=3$\\
			By the first equation the solution does not hold
		\subsection{Determine if the following matrices are in echoleon form or reduced echelon form}
			$\begin{bmatrix}
				1 & 0 & 0 \\
				0 & 1 & 0\\
				0 & 0 & 1 
			\end{bmatrix}$\\
			Reduced echelon form\\
			$\begin{bmatrix}
				1 & -3 & 4 & 7 \\
				0 & 1 & 2 & 2\\
				0 & 0 & 1  & 5
			\end{bmatrix}$\\
			Echelon form
	\section{Week 47}
		\subsection{
			
\end{document}
